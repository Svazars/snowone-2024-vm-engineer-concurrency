

\section{Как я вижу компиляторы}
\showTOC

\begin{frame}[t,fragile]{Классические оптимизации однопоточного кода}
\framesubtitle{Removing reads}


\begin{tabular}{p{0.5\textwidth} p{0.5\textwidth}}

\begin{minted}[fontsize=\small]{c}
static int a;
void foo_1() {
  while (true) {
    int tmp = a;
    if (tmp == 0) break;
    do_something_with(tmp);
  }
}
\end{minted}

&

\end{tabular}

\end{frame}


\begin{frame}[t,fragile,noframenumbering]{Классические оптимизации однопоточного кода}
\framesubtitle{Removing reads}

\begin{tabular}{p{0.5\textwidth} p{0.5\textwidth}}

\begin{minted}[fontsize=\small]{c}
static int a;
void foo_1() {
  while (true) {
    int tmp = a;
    if (tmp == 0) break;
    do_something_with(tmp);
  }
}
\end{minted}

&

\begin{minted}[fontsize=\small]{c}
// "optimized" version
void foo_2() {
  int tmp = a;
  if (tmp != 0)
    while (true) { 
      do_something_with(tmp); 
    }
}
\end{minted}
\end{tabular}

Имеет ли право компилятор уменьшить количество загрузок из памяти и переписать функцию?
\end{frame}


\begin{frame}[t, fragile]{Классические оптимизации однопоточного кода}
\framesubtitle{Godbolt}


\begin{tabular}{p{0.4\textwidth} p{0.5\textwidth}}


\begin{minted}[fontsize=\small]{c}
static int a;
void foo_1() {
  while (true) {
    int tmp = a;
    if (tmp == 0) break;
    do_something_with(tmp);
  }
}
\end{minted}

&

\end{tabular}
\end{frame}


\begin{frame}[t, fragile, noframenumbering]{Классические оптимизации однопоточного кода}
\framesubtitle{Godbolt}


\begin{tabular}{p{0.42\textwidth} p{0.5\textwidth}}


\begin{minted}[fontsize=\small]{c}
static int a;
void foo_1() {
  while (true) {
    int tmp = a;
    if (tmp == 0) break;
    do_something_with(tmp);
  }
}
\end{minted}

\texttt{x86-64 clang 16.0.0 -O2}\footnote{\tiny\url{https://godbolt.org/z/99j3erzaE}}

\texttt{x86-64 gcc 13.1 -O2}\footnote{\tiny\url{https://godbolt.org/z/fxzGEo1qf}}

&

\begin{minted}[fontsize=\small]{gas}
foo_1:                                  
    push    rbx
    mov     ebx, [a]
    test    ebx, ebx
    je      .LBB1_2
.LBB1_1:                       #<-|    
    mov     edi, ebx           #  |
    call    do_something_with  #  |
    jmp     .LBB1_1            #--|
.LBB1_2:
    pop     rbx
    ret
\end{minted}

\end{tabular}

\pause
\begin{tikzpicture}[remember picture,overlay]
\node[xshift=4.0cm,yshift=-4.0cm] at (current page.center) {\includegraphics[width=.2\textwidth]{production/bender-ass.png}};
\end{tikzpicture}

\end{frame}


\begin{frame}{Кто виноват и что делать?}

Очевидные преобразования однопоточных программ искажают поведение многопоточного кода.

\pause

Приходится изучать разные решения:

\pause
\begin{itemize}
	\item компиляторные барьеры
	\pause
  \item memory orderings (volatile, seq-cst, acquire-release...)\footnote<4->{\tiny\url{https://snowone.ru/2024/speakers/alantsov}}
  \pause
	\item примитивы синхронизации и их интеграцию с языком программирования % (threads cannot impl as library)
	\pause
	\item нативные отладчики и time-travel debugging
	\pause
	\item узкоспециализированные инструменты (model checking, deterministic scheduling...)
\end{itemize}

\end{frame}


% \begin{frame}{Выдумывание операций}
% 
% 2. Схлопывает вычислений. Добавляет вычисления. Выносит из секций, заносит в секции.
% Имеет право на совсем странные вещи. OOTA, что-то там еще
% 
% \end{frame}


\section{Как я вижу процессоры}
\showTOC

\begin{frame}[t]{Кругом враги}

Не только компиляторы (software) пытаются сломать ваше представление об исполнении программы в многопоточном контексте. 

\pause
Есть еще процессор и подсистема памяти (hardware). 

\pause
Которые умеют:
\begin{itemize}
    \pause
    \item Исполнять независимые инструкции одновременно (out-of-order execution)

    \pause
    \item Задействовать одни и те же ресурсы для исполнения логически независимых потоков (hyper-threading)
\end{itemize}
\end{frame}



\begin{frame}[t,noframenumbering]{Кругом враги}

Не только компиляторы (software) пытаются сломать ваше представление об исполнении программы в многопоточном контексте. 

Есть еще процессор и подсистема памяти (hardware). 

Которые умеют:
\begin{itemize}
    \item Исполнять независимые инструкции одновременно (out-of-order execution)

    \item Задействовать одни и те же ресурсы для исполнения логически независимых потоков (hyper-threading)

    \item Спекулировать\footnote{\tiny\url{https://en.wikipedia.org/wiki/Spectre_(security_vulnerability)}}\textsuperscript{,}\footnote{\tiny\url{https://en.wikipedia.org/wiki/Meltdown_(security_vulnerability)}}
    \begin{itemize}
        \item О предстоящих переходах (branch prediction)
        \item О требуемой памяти (cache prefetching)
        \item О результате вычислений (speculative execution)
        \item И многом другом
    \end{itemize}
\end{itemize}

\end{frame}


 \begin{frame}[fragile,t]{x86: Store buffering}
 
 
 \begin{minted}[fontsize=\small]{c}
                           int x, y;
 \end{minted}
 
 \begin{tabular}{p{0.5\textwidth} p{0.5\textwidth}}
 
 \begin{minted}[fontsize=\small]{c}
 void threadA() {
       x = 1;
   int a = y;
 }
 \end{minted}
 
 & 
 
 \begin{minted}[fontsize=\small]{c}
 void threadB() {                                   
         y = 1;                           
     int b = x;                           
 }                    
 \end{minted}
 \end{tabular}
 
 
 \pause
 
 \begin{tabular}{p{0.5\textwidth} p{0.5\textwidth}}
 \begin{minted}[fontsize=\small]{gas}
 # thread A
 mov [x] ,  1  # (A.1)
 mov EAX , [y] # (A.2)
 \end{minted}
 
 & 
 
 \begin{minted}[fontsize=\small]{gas}
 # thread B          
 mov [y] , 1  # (B.1) 
 mov EBX, [x] # (B.2) 
 \end{minted}
 \end{tabular}
 
 \end{frame}
 
 
 \begin{frame}[fragile,t,noframenumbering]{x86: Store buffering}
 
 \begin{tabular}{p{0.5\textwidth} p{0.5\textwidth}}
 \begin{minted}[fontsize=\small]{gas}
 # thread A
 mov [x] ,  1  # (A.1)
 mov EAX , [y] # (A.2)
 \end{minted}
 
 & 
 
 \begin{minted}[fontsize=\small]{gas}
 # thread B          
 mov [y] , 1  # (B.1) 
 mov EBX, [x] # (B.2) 
 \end{minted}
 \end{tabular}
 
 \pause
 Какие значения для \texttt{(EAX EBX)} допустимы?
 
 \texttt{\ \ \ \ \ \ \ \ \ \ \ \ \ \ \ \ \ \ (1 1)\ , (0 1)\ , (1 0)\ , (0 0)}
 
 \pause
 Варианты исполнения:
 \begin{itemize}
     \item \texttt{A.1 -> A.2 -> B.1 -> B.2}
     \item \texttt{\ \ \ \ \ \ \       B.1 -> A.2 -> B.2}
     \item \texttt{\ \ \ \ \ \ \ \ \ \ \ \ \ \              B.2 -> A.2}
     \item \texttt{B.1 -> A.1 -> A.2 -> B.2}
     \item \texttt{\ \ \ \ \ \ \ \ \ \ \ \ \ \              B.2 -> A.2}
     \item \texttt{\ \ \ \ \ \ \       B.2 -> A.1 -> A.2}
 \end{itemize}
 \end{frame}
 
 \begin{frame}[fragile,t,noframenumbering]{x86: Store buffering}
 
 \begin{tabular}{p{0.5\textwidth} p{0.5\textwidth}}
 \begin{minted}[fontsize=\small]{gas}
 # thread A
 mov [x] ,  1  # (A.1)
 mov EAX , [y] # (A.2)
 \end{minted}
 
 & 
 
 \begin{minted}[fontsize=\small]{gas}
 # thread B          
 mov [y] , 1  # (B.1) 
 mov EBX, [x] # (B.2) 
 \end{minted}
 \end{tabular}
 
 Какие значения для \texttt{(EAX EBX)} допустимы?
 
 \texttt{\ \ \ \ \ \ \ \ \ \ \ \ \ \ \ \ \ \ (1 1)\ , (0 1)\ , (1 0)\ , (0 0)}
 
 Варианты исполнения:
 \begin{itemize}
     \item \texttt{A.1 -> A.2 -> B.1 -> B.2}                            : \texttt{(0, 1)}
     \item \texttt{\ \ \ \ \ \ \       B.1 -> A.2 -> B.2}               : \texttt{(1, 1)}
     \item \texttt{\ \ \ \ \ \ \ \ \ \ \ \ \ \              B.2 -> A.2} : \texttt{(1, 1)}
     \item \texttt{B.1 -> A.1 -> A.2 -> B.2}                            : \texttt{(1, 1)}
     \item \texttt{\ \ \ \ \ \ \ \ \ \ \ \ \ \              B.2 -> A.2} : \texttt{(1, 1)}
     \item \texttt{\ \ \ \ \ \ \       B.2 -> A.1 -> A.2}               : \texttt{(1, 0)}
 \end{itemize}
 \end{frame}
 
 \begin{frame}[fragile,t,noframenumbering]{x86: Store buffering}
 
 \begin{tabular}{p{0.5\textwidth} p{0.5\textwidth}}
 \begin{minted}[fontsize=\small]{gas}
 # thread A
 mov [x] ,  1  # (A.1)
 mov EAX , [y] # (A.2)
 \end{minted}
 
 & 
 
 \begin{minted}[fontsize=\small]{gas}
 # thread B          
 mov [y] , 1  # (B.1) 
 mov EBX, [x] # (B.2) 
 \end{minted}
 \end{tabular}
 
 Какие значения для \texttt{(EAX EBX)} допустимы?
 
 \texttt{Ответ:\ \ \ \ \ \ \ \ \ \ \ \ (1 1)\ , (0 1)\ , (1 0)}
 
 Варианты исполнения:
 \begin{itemize}
     \item \texttt{A.1 -> A.2 -> B.1 -> B.2}                            : \texttt{(0, 1)}
     \item \texttt{\ \ \ \ \ \ \       B.1 -> A.2 -> B.2}               : \texttt{(1, 1)}
     \item \texttt{\ \ \ \ \ \ \ \ \ \ \ \ \ \              B.2 -> A.2} : \texttt{(1, 1)}
     \item \texttt{B.1 -> A.1 -> A.2 -> B.2}                            : \texttt{(1, 1)}
     \item \texttt{\ \ \ \ \ \ \ \ \ \ \ \ \ \              B.2 -> A.2} : \texttt{(1, 1)}
     \item \texttt{\ \ \ \ \ \ \       B.2 -> A.1 -> A.2}               : \texttt{(1, 0)}
 \end{itemize}
 \end{frame}
 
 \begin{frame}[fragile,t,noframenumbering]{x86: Store buffering}
 
 \begin{tabular}{p{0.5\textwidth} p{0.5\textwidth}}
 \begin{minted}[fontsize=\small]{gas}
 # thread A
 mov [x] ,  1  # (A.1)
 mov EAX , [y] # (A.2)
 \end{minted}
 
 & 
 
 \begin{minted}[fontsize=\small]{gas}
 # thread B          
 mov [y] , 1  # (B.1) 
 mov EBX, [x] # (B.2) 
 \end{minted}
 \end{tabular}
 
 Какие значения для \texttt{(EAX EBX)} допустимы?
 
 \only<1>{\texttt{Ответ:}}\only<2->{{\color{red}\texttt{Правильный ответ:}}} \only<1>{\texttt{\ \ \ \ \ \ \ \ \ \ \ }} \texttt{(1 1)\ , (0 1)\ , (1 0)}
 \pause
 {\color{red} \texttt{, (0 0)}}
 
 \pause
 Процессор иногда переупорядочивает записи и чтения.
% \pause
% В данном случае изменился наблюдаемый другими процессорами порядок store и load операций\footnote<4->{\tiny\url{https://habr.com/ru/company/JetBrains-education/blog/523298/}}\textsuperscript{,}\footnote<4->{\tiny\url{https://diy.inria.fr/doc/SB.litmus}}\textsuperscript{,}\footnote<4->{\tiny\url{https://www.cl.cam.ac.uk/~pes20/weakmemory/acm.pdf}}.
 
 \pause
 Вывод: порядок инструкций в машинном коде $\neq$ порядок наблюдаемых эффектов этих инструкций.

 \pause

\begin{tikzpicture}[remember picture,overlay]
\node[xshift=4.0cm,yshift=-3.5cm] at (current page.center) {\includegraphics[width=.3\textwidth]{production/fry-pain-head.png}};
\end{tikzpicture}

\end{frame}
 
 
 \begin{frame}[fragile]{arm64: Independent Reads of Independent Writes}
 
 
 \begin{tabular}{p{0.2\textwidth} p{0.2\textwidth} p{0.2\textwidth} p{0.2\textwidth}}
 \begin{minted}[fontsize=\small]{python}
 thread1
   x = 1
 \end{minted}
 
 & 
 
 \begin{minted}[fontsize=\small]{python}
 thread2
   y = 1
 \end{minted}
 
 &
 
 \begin{minted}[fontsize=\small]{python}
 thread3
  r1 = x
  r2 = y 
 \end{minted}
 
 &
 
 \begin{minted}[fontsize=\small]{python}
 thread4
  r3 = y
  r4 = x
 \end{minted}
 \end{tabular}
 
 
 \pause
 Может ли быть так, что \texttt{(r1 = 1, r2 = 0, r3 = 1, r4 = 0)}?
 
 %\pause
 %Могут ли потоки 3 и 4 увидеть изменения в разном порядке?
 
 \pause
 При условии, что переупорядочивание чтений не происходит.
 
 \pause
 \begin{itemize}
 \item На \texttt{x86} или \texttt{x86\_64} (TSO): нет
 
 \pause
 \item На \texttt{ARM} или \texttt{POWER}: да\footnote<5->{\tiny A Tutorial Introduction to the ARM and POWER Relaxed Memory Models, section 6.1}
 \end{itemize}
 
 \pause
 Записи могут "доехать"\ до других процессоров в разном порядке.
 
 \pause
 У каждого процессора своя временная шкала и некоторое видение окружающего мира. Возможно, отличающееся от других процессоров.
 
 \pause
 Вывод: нельзя рассматривать "запись в ячейку памяти"\ как точку на единой временной шкале\footnote<8->{\tiny The Art of Multiprocessor Programming by Maurice Herlihy \& Nir havit, Chapter 3 "Concurrent Objects"}.


\pause

\begin{tikzpicture}[remember picture,overlay]
\node at (current page.center) {\includegraphics[width=.6\textwidth]{production/fry-crazy.png}};
\end{tikzpicture}

 \end{frame}

 
 
 \begin{frame}[t]{Почему так сложно?}
 
 \begin{itemize}
  \item порядок инструкций в машинном коде $\neq$ порядок наблюдаемых эффектов этих инструкций 
  \pause
  \item нельзя рассматривать "чтение/запись в ячейку памяти"\ как точку на единой временной шкале
  \pause
  \item у каждого процессора свои правила
 \end{itemize}
 
 \pause

 \begin{center}
 \includegraphics[width=.3\textwidth]{production/bender-kill.jpg}
 \end{center}

 \end{frame}


\begin{frame}[t,noframenumbering]{Почему так сложно?}
 
 \begin{itemize}
  \item порядок инструкций в машинном коде $\neq$ порядок наблюдаемых эффектов этих инструкций 
  \item нельзя рассматривать "чтение/запись в ячейку памяти"\ как точку на единой временной шкале
  \item у каждого процессора свои правила
 \end{itemize}
 
 Почему вообще хоть кто-то пользуется ARM/POWER/RISC-V и другими процессорами со слабой моделью памяти? 
 \pause
 \begin{itemize}
   \item производительность
   \pause
   \item Производительность!
   \pause
   \item энергосбережение :)
 \end{itemize}
 
 \end{frame}


% \begin{frame}{Мастера перестановок}
% 
% 1. Переставляет, зараза многое и не так. И барьеры неинтуитивные и много их разных, на разных платформах.
% Литмусы эти.
% 
% \end{frame}
% 
% 
% \begin{frame}{Непредсказуемая скорость работы}
% 
% 2. ВСе эти непредсказуемые performacne модели памяти. То быстро, то медленно, то false sharing, то префетч отвалился, то вымывание, то minor page fault
% Кода компилятор сгенерил в 3 раза больше, чем ручной ассемблер и при этом стало быстрее. ой границу цикла ровнять. Или колокацию. Или не надо. А-а-а
% 
% \end{frame}

